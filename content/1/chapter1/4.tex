指标的概念是性能概念的基础,其总是隐含着可能性和必要性:如果说“我们有一个指标”,就意味着有一种量化和衡量的方法,而找出指标值的唯一方法就是测试。

衡量性能的重要性怎么强调都不为过,性能的第一定律就是永远不要去猜测性能。本书的下一章将专注于性能测试、测试工具、如何使用它们,以及如分析结果。

不幸的是,很多时候会对性能进行猜测。还有一些过于笼统的语句,如“避免在C++中使用虚函数,它们很慢。”其问题在于精确,这里并没有说明虚函数相对于非虚函数慢多少的度量。作为读者的练习,这里有几个答案可供选择,都已经量化:

\begin{itemize}
\item 虚函数慢100\%
\item 虚函数慢15~20\%
\item 虚函数对程序没什么影响
\item 虚函数快10~20\%
\item 虚函数慢100倍
\end{itemize}

哪个是正确答案呢?如果你选择了这些答案中的任何一个,恭喜你,选择了正确的答案。没错,在特定的环境和特定的上下文中,这些答案都是正确的(要了解原因,需要等到第9章,高性能C++)。

通过接受直觉或猜测性能几乎是不可能的事实,我们会有落入另一个陷阱的风险:用它作为借口编写低效的代码,再“稍后进行优化”,因为我们不猜测性能。诚然,后一条格言可能会走得太远,就像“不要过早优化”一样。

性能指标不能在后期添加到程序中,所以在最初的设计和开发中应该进行考虑。与其他设计目标一样,性能因素和目标也应在设计阶段占有一席之地。这些早期的目标和永远不要猜测绩效的规则之间存在着明显的矛盾,我们必须找到合适的折衷方案,描述设计阶段想要实现的性能目标是一个好办法。虽然提前知晓了最佳优化是不可能的,但可以确定可能导致后续优化变得非常困难,甚至不可实施的设计决策。

在程序开发过程中也会出现同样的情况:花很长时间优化一个每天只调用一次、只需要一秒钟的函数是愚蠢的行为。另一方面,将这些代码封装到一个函数中则非常明智。因此,如果随着程序的发展,使用模式发生了变化,则可以以后再对其进行优化,而无需重写其余部分的代码。

另一种描述“不提前优化规则”的局限性,是通过“是”来进行限定,但也不要过于悲观。认识到两者之间的差异需要良好的设计实践/知识,以及对编程的不同方面的理解,从而才能获得高性能。

那么,作为一名开发人员/编程者,为了精通开发高性能应用程序,需要学习和理解什么呢?下一节,我们将从这些目标开始,然后详细讨论每一个。
















