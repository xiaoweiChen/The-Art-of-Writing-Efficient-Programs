是什么使得程序高性能?可以说“效率”。首先,这并不总是正确的(尽管它经常正确);其次,它回避了一个问题,因为下一个问题就是:是什么使得程序有效率?为了写出高效或高性能的程序,需要学习什么?让我们制作一个列表,来看看都需要哪些技能和知识:

\begin{itemize}
\item 选择正确的算法
\item 有效利用CPU资源
\item 高效的使用内存
\item 避免不必要的计算
\item 有效地使用并发和多线程
\item 有效地使用编程语言,避免效率低下
\item 衡量性能和分析结果
\end{itemize}

实现高性能的最重要因素是选择一个好的算法,我们不能通过优化实现来“修复”算法的缺点。对于算法好坏的讨论,超出了本书的范畴。当然,算法是具体问题的解法,你必须自己做研究,找出解决问题的最优解法。

另一方面,实现高性能的方法和技术在很大程度上与问题无关。当然,它们确实依赖于性能指标,例如:实时系统的优化是高度特化领域的问题。本书中,我们主要关注高性能计算上的性能指标:尽可能快地进行计算。

为了成功地完成这项任务,我们必须学会尽可能多地使用可用的计算硬件。这个目标有空间和时间的两部分组成:空间方面,我们谈论的是使用更多的晶体管,处理器拥有如此巨大的数量。处理器正在变得更大,甚至更快。增加的面积是为了能提供新的计算能力。就时间而言,我们应该使用尽可能多的硬件。无论如何,如果计算资源空闲,那么它们对我们来说是没有用的,所以我们的目标是避免这种情况。与此同时,忙碌也会没有回报,我们要避免做任何不需要的事情。我们要从哪里下手呢?有很多方法可以让程序执行不需要的计算。

本书中,我们将从单处理器开始,并学习如何有效地使用它的计算资源。然后,我们将扩展视图,不仅包括处理器,还包括内存。当然,我们也会考虑同时使用多个处理器的情况。

有效地使用硬件只是高性能程序的特性之一:高效地完成本来可以避免的工作对我们没有任何帮助。高效工作的关键是有效地使用编程语言,我们的例子中是使用的是C++(我们对硬件的大部分了解可以应用到任何语言,但是一些语言优化技术是C++特有的)。此外,编译器位于我们编写的语言和使用的硬件之间,因此我们必须学习如何使用编译器来生成最有效的代码。

最后,量化我们刚刚列出的目标成功程度的方法就是进行测试:我们使用了多少CPU资源?花了多少时间等待内存?增加另一个线程是否可以获得性能增益?等等。获得良好的量化性能数据并不容易,这需要对测量工具有更细节的理解,分析结果往往会更难。

可以期待从这本书中学习以上的技能。我们将学习硬件架构,以及隐藏在一些编程语言特性背后的东西,以及如何像编译器那样查看代码。技能固然重要,但更重要的是理解为什么会以这种方式运行。计算硬件经常发生变化,语言也在不断发展,你也可以为编译器发明了新的优化算法。因此,任何这些领域的特定知识的保质期都不会太久。无论如何,了解使用特定处理器或编译器的最佳方法,而且还了解我们获得这些知识的方法,那么就可以重复这个过程,从而进行更深入的学习。
































