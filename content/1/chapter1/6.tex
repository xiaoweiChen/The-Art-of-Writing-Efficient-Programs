这一导论性章节中,我们讨论了为什么现代计算机的原始计算能力飞速发展,人们对软件性能和效率的兴趣却在上升。因此,为了理解限制性能的因素,以及如何克服它们,我们需要回到计算的基本要素上来。理解计算机和程序是如何在较低水平上工作的:了解硬件并有效地使用,理解并发性,理解C++语言特性和编译器优化,以及其对性能的影响。

这种底层知识必然是详细和具体的,但我们有一个计划:当了解处理器或编译器的具体情况后,我们也将了解得出这些结论的过程。在更深层次上,本书更是关于如何学习的。

我们进一步了解到,如果不定义衡量性能的标准,性能的概念就没有意义。需要根据特定的指标来评估性能,这意味着任何关于性能的工作都由数据和指标驱动。下一章,我们来了解一下如何对性能进行测试。