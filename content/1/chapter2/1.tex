首先,需要一个C++编译器。本章的示例使用GCC或Clang编译器,并在Linux上进行编译。所有的Linux发行版都会将GCC作为常规安装的一部分,发行版中可能有较新的编译器版本。Clang编译器可以通过LLVM项目\url{http://llvm.org/}获得。Windows上,Microsoft Visual Studio是最常用的编译器,当然GCC和Clang也可以使用。

其次,需要一个分析工具。本章中,我们将使用Linux的perf性能分析器,其在大多数Linux发行版上都已安装(或可用于安装)。文档的地址:\url{https://perf.wiki.kernel.org/index.php/Main_Page}。

还会演示另一个分析器的使用,来自于谷歌性能工具集(GperfTools)的CPU分析器,地址为:\url{https://github.com/gperftools/gperftools}(同样,可以通过其源码进行安装)。

还有是许多其他可用的分析工具,有免费的,也有商业的。它们以不同的方式展示了相同类型的信息,但会提供许多不同的分析选项进行呈现。通过本章的示例,可以了解什么是分析工具,以及可能存在的限制。有着良好纪律性的开发者,会对使用的工具细节进行详细的了解。

最后,使用微基准测试工具。本章中使用了\url{https://github.com/google/benchmark}谷歌基准库,需要自己下载和安装(即使与Linux发行版一起安装,版本也很可能过时),请按照页面上的说明进行安装。

安装了所有必要的工具后,就可以进行第一次性能测试了。

本章代码地址: \url{https://github.com/PacktPublishing/The-Art-of-Writing-Efficient-Programs/tree/master/Chapter02}。



