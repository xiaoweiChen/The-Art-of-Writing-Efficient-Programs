Whether writing a new high-performance program or optimizing an existing one, one of the first tasks set before you will be to define the performance of the code in its current state. Your success will be measured by how much you can improve its performance. Both of these statements imply the existence of a performance metric, something that can be measured and quantified. One of the more interesting outcomes of the last chapter was the discovery that there isn't even a single definition of performance that fits every need: what you measure when you want to quantify performance depends on the nature of the problem you're working on.

But there is much more to the measurements than simply defining the goals and confirming success. Every step of your performance optimization, whether of existing code or new code you're just writing, should be guided and informed by measurements.

The first rule of performance is Never guess about performance, and it is worth dedicating the first section of this chapter to the goal of convincing you to take this rule to heart without questions or doubts. After shattering your faith in your intuition, we have to give you something else to stand on instead: the tools and approaches for measuring and learning about performance.

In this chapter, we're going to cover the following main topics:

\begin{itemize}
\item Why performance measurements are essential
\item Why all performance-related decisions must be driven by measurements and data
\item How to measure the performance of real programs
\item What is benchmarking, profiling, and micro-benchmarking of programs, and how to use them to measure performance
\end{itemize}

