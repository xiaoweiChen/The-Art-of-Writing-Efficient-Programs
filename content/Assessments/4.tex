\begin{enumerate}
\item 
现代CPU甚至比最好的内存都要快,访问内存中随机位置的延迟是几纳秒,这足以让CPU执行数十次操作。即使在流访问中,内存的总带宽也不足以以相同的速度为CPU提供数据,以执行计算。

\item 
内存系统包括CPU和主存之间的缓存层次结构,因此影响速度的第一因素是数据集的大小,这最终决定了数据是否适合缓存。对于给定的大小,内存访问模式是关键,若硬件可以预测下一次访问,就可以通过在请求数据之前开始将数据传输到缓存来隐藏延迟。

\item 
低效的内存访问是显式的性能数据文件或计时器输出,对于具有良好数据封装的模块化代码来说也是如此。在计时分析没有进行统计和分析的地方,缓存有效性可能会在整个代码中显示低效访问的数据。

\item
使用较少内存的优化可能会提高内存性能,因为更多的数据适合于缓存。但对大量数据的顺序访问可能比对少量数据的随机访问要快,较小的数据适合L1缓存,或者适合L2缓存。直接针对内存性能的优化通常采用数据结构优化的形式,主要目的是避免随机访问和间接内存访问。除此之外,还必须改变算法,将内存访问模式改为更友好的缓存模式。

\end{enumerate}