\begin{enumerate}
\item 
In many domains, the size of the problems grows as fast as or even faster than the available computational resources. As computing becomes more ubiquitous, heavy workloads may have to be executed on processors of limited power.

\item 
Single-core processing power largely stopped increasing about 15 years ago, and the advances in processor design and manufacturing largely translate into more processing cores and a large number of specialized computing units. Making the best use of these resources does not happen automatically and requires an understanding of how they work.

\item 
Efficiency refers to using more of the available computational resources more of the time and not doing any unnecessary work. Performance refers to meeting specific targets that depend on the problem the program is designed to solve.

\item
In different environments, the definition of performance may be completely different: the raw speed of the computation may be all that matters in a supercomputer, but it is not relevant in an interactive system as long as the system is faster than the person interacting with it.

\item
Performance must be measured; the proof of success or the guidance to the causes of the failure is in the quantitative measurement results and their analysis.
	
\end{enumerate}