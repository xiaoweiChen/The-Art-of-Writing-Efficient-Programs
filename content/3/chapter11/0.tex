This chapter has a dual focus. On the one hand, it explains the dangers of the kinds of undefined behavior that programmers often ignore when attempting to squeeze the most performance from their code. On the other hand, it explains how to take advantage of the undefined behavior to improve performance and how to properly specify and document such situations. Overall, the chapter offers a somewhat unusual but more relevant way to understand the issue of the undefined behavior compared to the usual "anything can happen." 

In this chapter, we will cover the following topics:

\begin{itemize}
\item 
Understanding undefined behavior and why it exists

\item 
Understanding the truth versus the myths about undefined behavior

\item 
How to take advantage of undefined behavior

\item 
Memory bandwidth and latency

\item 
Learning the connection between undefined behavior and efficiency and how to exploit it
\end{itemize}

You will learn to recognize undefined behavior when it is encountered in (somebody else's) code and understand how undefined behavior is related to performance. This chapter also teaches you how to use undefined behavior for good by intentionally allowing it, documenting it, and placing safeguards around it.