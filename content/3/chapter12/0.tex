This chapter reviews all the performance-related factors and features we have learned in this book and explores the subject of how the knowledge and the understanding we have gained should influence the design decisions we make when developing a new software system or rearchitecting an existing one. We will see how the design decisions impact the performance of the software systems, learn how to make performance-related design decisions in the absence of detailed data, as well as reviewing the best practices for designing APIs, concurrent data structures, and high-performance data structures to 

avoid inefficiencies. We will explore the following subjects:

\begin{itemize}
\item 
Interaction between the design and performance

\item 
Design for performance

\item 
API design considerations

\item 
Design for optimal data access

\item 
Performance trade-offs

\item 
Making informed design decisions
\end{itemize}

You will learn how to treat good performance as one of the design goals from the beginning and how to design high-performance software systems in ways that ensure that efficient implementation does not become a struggle against the fundamental architecture of the program.